\documentclass{report}
\usepackage{graphicx}
\usepackage{listings}
\usepackage{xcolor}
\usepackage{array}
\usepackage[a4paper, total={6in, 10in}]{geometry}
\usepackage{amssymb}
\usepackage{amsmath}
\usepackage{tikz}


\definecolor{codegreen}{rgb}{0,0.6,0}
\definecolor{codegray}{rgb}{0.5,0.5,0.5}
\definecolor{codepurple}{rgb}{0.58,0,0.82}
\definecolor{backcolour}{rgb}{0.95,0.95,0.92}

\lstdefinestyle{mystyle}{
	backgroundcolor=\color{backcolour},   
	commentstyle=\color{codegreen},
	keywordstyle=\color{magenta},
	numberstyle=\tiny\color{codegray},
	stringstyle=\color{codepurple},
	basicstyle=\ttfamily\footnotesize,
	breakatwhitespace=false,         
	breaklines=true,                 
	captionpos=b,                    
	keepspaces=true,                 
	numbers=left,                    
	numbersep=5pt,                  
	showspaces=false,                
	showstringspaces=false,
	showtabs=false,                  
	tabsize=2
}
\lstset{style=mystyle}
\graphicspath{ {images/} }
\title{Design II\\ HoGent}
\author{JeeVeeVee}
\date{2020/2021}
\begin{document}
	\maketitle
   	\tableofcontents
   	\chapter{Factory pattern}
   		The principle of a factory is that you take the code for creating an object, and move it to another object which will have as single responsibility creating these objects. 
   		\section{Simple Factory}
   			Often works with a simple switch statement. The factory has a client who 'orders' the object from the factory. The create-method is often static.
   		\section{Factory Method}
   			There is an abstract class with an abstract method, the factory method. The sub-classes implement this factory method, this way, you don't have to outsource the creation of the objects. The sub-classes decide how the method is implemented. In this case, the class hierarchies are parallel : both contain abstract classes which are extended into concrete classes which contain specific implementations.
   			\begin{center}
   				\includegraphics[scale=0.3]{factory_method}
   			\end{center}
   			The factory method relies on the \textbf{dependency inversion principle} which states that it is beter to be dependent on abstractions rather then to be dependent on concrete classes.
   			\pagebreak
   		\section{Abstract Factory}
   			An abstract factory supplies an interface for a set of products. The code is designed so that we use a factory to create the products, this way the client code stays the same, we do get multiple implementations of our products.
   			\begin{center}
   				\includegraphics[scale=0.3]{abstract_factory}
   			\end{center}
   	\chapter{Iterator and Composite pattern}
   		\section{Iterator patters}
   			You want the client to access 2 different implementations with the same functionality. To achieve this, you have to create some kind of translator in the middle who can communicate with both of the implementations. This way the differences in implementation are invisible to the client. The iterator pattern feels a lot like a facade...
   			\begin{center}
   				\includegraphics[scale=0.3]{iterator}
   			\end{center}
   			The iterator pattern is build on the \textbf{single responsibility principle} which states that there should only reason for a class to change.
   			\pagebreak
   		\section{Composite Pattern}
   			The composite patters allows you to represent a part-whole hierarchy as a tree structure, enabling you to process individual objects or a group of objects in a uniform way. In short, the composite allows us to represent the following situation : 
   			\begin{center}
   				\includegraphics[scale=0.3]{composite_irl}
   			\end{center} 
   			By this UML: 
   			\begin{center}
   				\includegraphics[scale=0.3]{composite_uml}
   			\end{center} 
   			The Composite defines the behavior of components with child components and contains children. The Leaf defines the behavior of elements in the hierarchy, it has no children. The Component is the interface for objects in the hierarchy. This does mean that some methods in both Leaf and Composite will throw UnsupportedException since they don't makes sense otherwise.
	\chapter{Builder}
		The builder pattern is used to encapsulate (hide) the construction logic, allowing for complex objects to be constructed in incremental steps. Some classes have lot of options and multiple constructors, you could solve this by using setters, but this doesn't guarantee the correct order which is sometimes pretty important. Another solution is a Builder class, this way we can define a fixed order of steps, disadvantage of this method is that we can only build 1 specific type of object (the one with that particular order). To make our design more flexible, we need a flexible data structure which can represent different objects. We can achieve this flexibility by using an Iterator.
   		\begin{center}
   			\includegraphics[scale=0.5]{builder}
   		\end{center} 
   		The Builder class specifies an abstract interface for creating parts of the Product object. The Concrete Builder implements the Builder interface. The Director class builds the complex object, using a Builder interface. Product represents the complex object that is being built.
   		The advantages are : 
   		\begin{itemize}
   			\item encapsulates how a complex object is being built 
   			\item allows for creating an object in multiple steps, where there is only 1 step if you use a factory.
   			\item hides the internal representation of the product form the client
   			\item the product implementations can change at all times since the client only accesses the abstract interface it implements.
   		\end{itemize}
	\chapter{Command Pattern}
		The Command pattern encapsulates a request as an object, thereby letting you parameterize other object with different requests, queue or log requests, and support undoable operations. 
   		\begin{center}
   			\includegraphics[scale=0.4]{command}
   		\end{center} 
   		The Client is responsible for creating a ConcreteCommand, The receiver knows how to preform the work needed to carry out the request, this could be any class. The Invoker holds the command and at some point, it will ask the command to carry out a request by calling the .execute() method.
   		\\
   		The Command pattern decouples an object making a request from the one that knows how to preform it. This pattern feels a lot like State...
   	\chapter{Template Method Pattern}
   		The Template Method Pattern defines the skeleton of an algoritm in a method, leaving some steps to subclasses. The pattern relies on subclasses to define some steps of an algoritm without allowing them to change the general structure of the algoritm(because the templateMethod is final). 
   		\begin{center}
   			\includegraphics[scale=0.6]{template_method}
   		\end{center} 
   		You can use a hook to overwrite a method, if you would want to. 
   		\\
   		This pattern uses the \textbf{Hollywood principle} which states "Don't call us, we'll call you". This mean the high-level component decides when and how the low-level component participates, a low-level component however, never invokes a high-level component.
	\chapter{Adapter}
		An Adapter is used to link an existing system to library classes.This way Adapter allows classes to work together even if they have incompatible interfaces.
		\begin{center}
			\includegraphics[scale=0.3]{adapter}
		\end{center} 
\end{document}